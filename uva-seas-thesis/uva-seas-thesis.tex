\documentclass[10pt]{article}

\usepackage[acronym]{glossaries}
\usepackage{listings}
\usepackage{mdwlist}
\usepackage{url}
\usepackage[svgnames]{xcolor}

\loadglsentries{glossary}

\lstset{%
  language=[LaTeX]TeX,%
  basicstyle=\small\ttfamily,%
  keywordstyle=\color{Blue},%
  commentstyle=\color{Green}\ttfamily,%
  stringstyle=\rmfamily,%
  numberstyle=\scriptsize,%
  stepnumber=5,%
  showstringspaces=false,%
  breaklines=true,%
  frameround=ftff,%
  frame=single%
}

\newcommand{\class}[1]{#1}
\newcommand{\package}[1]{#1}
\newcommand{\option}[1]{#1}
\newcommand{\environment}[1]{\texttt{#1}}
\newcommand{\command}[1]{\texttt{\textbackslash#1}}
\newcommand{\argument}[1]{\texttt{#1}}
\newcommand{\default}[1]{\emph{#1}}

\newcommand{\thispackage}{\package{uva-seas-thesis}}

\title{\thispackage}
\author{
  Joel Coffman\\
  \url{jcoffman@cs.virginia.edu}
}

\begin{document}
\maketitle

The \thispackage\ \LaTeX\ package provides formatting and stylistic commands that ensure documents adhere to the University of Virginia's \gls{SEAS} and Print Services guidelines.
Like other \LaTeX\ packages, \thispackage\ forces the author to focus on writing content instead of stalling by endlessly changing the document's formatting.

%-------------------------------------------------------------------------------
\section{Introduction}\label{section:introduction}
When I started writing my master's thesis in 2009, I discovered that there was no \LaTeX\ package that provided a concise set of commands for formatting my thesis according to the guidelines of \gls{SEAS}.
The computer science graduate student group referred students to David Coppit's paper template, which faithfully formatted documents according to the aforementioned guidelines, but I thought it lacked the elegance of a complete \LaTeX\ package.
Thus, I started working on my own \LaTeX\ package to provide similar functionality without exposing the author to any of the low-level commands (like formatting the signatures page).

The result is the \thispackage\ package, which is designed to be easier to use than other classes or packages, many of which (at least for \gls{SEAS}) mix content and presentation.
This package is designed to be interoperable with a variety of \LaTeX\ classes (e.g., \class{book}), which control most of the actual formatting decisions.
This documentation is designed to explain options, environments, and commands provided by this package.
It is only fair to acknowledge that many of the package options and this document itself are patterned after Andy Buckley's \class{hepthesis} class, which partly inspired me to begin writing this package.

If you use this package to format you thesis, I'd enjoy receiving an email so I know others appreciate this work.
You're also more than welcome to include comments and suggestions so I can improve future versions of this package.

%-------------------------------------------------------------------------------
\section{Requirements}\label{section:requirements}
\LaTeXe\ is obviously required for typesetting documents using this package.
The required packages are
\begin{description*}
  \item[\package{geometry}] sets appropriate page margins
  \item[\package{mfirstuc}] capitalizes the first letter of words when necessary
  \item[\package{setspace}] double spacing
\end{description*}
Additional packages are required depending on the package options.
The package options and the packages they load are listed below.
\begin{description*}
  \item[\option{ams}] \package{amsmath}
  \item[\option{booktabs}] \package{booktabs}
  \item[\option{draft}] \package{type1cm}, \package{draftwatermark}, \package{lineno}
  \item[\option{hyper}] \package{hyperref}, \package{hypcap}
\end{description*}

%-------------------------------------------------------------------------------
\section{Package Options}\label{section:options}
This section describes all the options provided by \thispackage.
Default values for options are denoted by \default{italics}.

\subsection{\default{\option{ams}} \textbar\ \option{noams}}
Loads the \package{amsmath} package to improve the typesetting of mathematical formulas.

\subsection{\option{bind} \textbar\ \default{\option{nobind}}}
Sets the margins suitable for binding or ordinary viewing.
The bind option increases the size of the inner (i.e., left, unless \option{twoside} is specified) margin whereas nobind has equal margins.

\subsection{\default{\option{booktabs}} \textbar\ \option{nobooktabs}}
Loads the \package{booktabs} package to enhance the quality of tables.

\subsection{\option{compliantcopyright}}
Creates the copyright notice stipulated by the University of Virginia print services.
Without this option, a more traditional copyright notification is used.

\subsection{\option{draft}}
Prints ``DRAFT" diagonally across the page and adds line numbers in the margin of each page.

\subsection{\option{hidefront} \textbar\ \option{hideback} \textbar\ \option{hidefrontback}}
Hides the respective sections of the thesis for commands defined by this package.
These options are particularly useful for drafts because they eliminate most of the boilerplate sections (e.g., copyright page, table of contents, list of tables) that aren't necessary for drafts.
Note that these options do \emph{not} remove all material from the respective sections, just material created by this package's commands.

\subsection{\option{hyper}}
Loads the \package{hyperref} package to create hyperlinks in the PDF document and the \package{hypcap} package to correct location of the link target.

\subsection{\option{final}}
Creates a ``final'' (i.e., suitable for binding) version of the document.
Specifying this option implies the \option{bind} option and disables the \option{draft}, \option{hidefront}, \option{hideback}, \option{hidefrontback}, \option{hyper}, \option{openany}, and \option{twoside} options.
Note that this package cannot override options specified to the document class (e.g., \option{openany} or \option{twoside}); these global options should be removed to guarantee the desired result is achieved.

\subsection{\option{openany}}
This option redefines the \texttt{\textbackslash cleardoublepage} command as \texttt{\textbackslash clearpage} so chapters, etc. may start on odd or even numbered pages.

\subsection{\option{twoside}}
If the bind option is specified, this option assumes pages will be printed doublesided (i.e., the inner margin alternates between the left and right sides of the page).

%-------------------------------------------------------------------------------
\section{Environments and Commands}

\subsection{\command{degree\{text\}}}
Specifies the degree being sought.
The text argument should be ``Master of Science,'' ``Master of Engineering,'' or ``Doctor of Philosophy.''
If this command does \emph{not} appear in the document preamble, ``Doctor of Philosophy'' is assumed.

\begin{lstlisting}[gobble=2,float=t,caption={Sample document preamble showing the \command{degree}, \command{program}, \command{documenttype}, \command{graduationmonth}, and \command{graduationyear} commands.}]
  \author{John Doe}
  \title{A Conceptual Object-Oriented Interface for an Integrated Logical Toolkit}
  
  \degree{Doctor of Philosophy}
  \program{Computer Science}
  \documenttype{dissertation}
  
  \graduationmonth{May}
  \graduationyear{2096}
\end{lstlisting}

\subsection{\command{program\{text\}}}
Specifies the program of study (e.g., the department).
This command \emph{must} appear in the document preamble.

\subsection{\command{documenttype\{text\}}}
Specifies the type of document (i.e., thesis or dissertation).
The text argument should be all lower case
If this command does \emph{not} appear in the document preamble, ``thesis'' is assumed.

\subsection{\command{graduationmonth\{month\}}}
Identifies the month your degree will be awarded (i.e., May, August, or January), \emph{not} the date when you complete or defend your work.
This command \emph{must} appear in the document preamble.

\subsection{\command{graduationyear\{year\}}}
Identifies the year your degree will be awarded.
This command \emph{must} appear in the document preamble.

\subsection{\command{copyrightpage}}
Adds a copyright page to your document.
The copyright page should immediately follow the title page.

\subsection{\environment{signatures}}
Creates a signatures page (aka approval sheet) for the document.
The signatures page is where you, your committee, and the dean sign your document.

\begin{lstlisting}[gobble=2]
  \renewcommand{\signaturestyle}[1]{\LARGE #1}
  \begin{signatures}
    \signature[Richard Miles]{Richard Miles, Adviser}
    \signature[Fred Bloggs]{Fredd Bloggs, Committee Chair}
    ...
    
    \dean{John Stiles}
  \end{signatures}
\end{lstlisting}

\subsubsection{\command{signaturestyle\{commands\}}}
Redefine this command to control how signatures appear in electronic versions.
This command must be redefined \emph{before} the environment to correctly typeset the author's name.

\subsubsection{\command{signature[name]\{person\}}}
Creates a signature line for the specified individual.
The \argument{person} argument specifies the person and optionally that individuals position on the committee (e.g., adviser or chair).
The \argument{name} argument specifies the text that should appear on the signature line for electronic versions of the thesis.

\subsubsection{\command{dean\{name\}}}
Specifies the name of the dean of the School of Engineering and Applied Science.

\subsection{\environment{abstract}}
Defines an abstract environment (not part of the traditional \LaTeX\ book class) for the document's abstract.

\begin{lstlisting}[gobble=2]
  \begin{abstract}
    This research problem is very challenging...
  \end{abstract}
\end{lstlisting}

\subsection{\environment{acknowledgments}}
Defines an acknowledgments environment for thanking everyone.

\begin{lstlisting}[gobble=2]
  \begin{acknowledgments}
    To everyone who's helped me during the past seven years...
  \end{acknowledgments}
\end{lstlisting}

\subsection{\command{texorpdfstring\{\TeX string\}\{PDFstring\}}}
The \package{hyperref} package cannot handle fragile commands in moving references (e.g., chapter or section titles).
This command allows a \TeX\ and PDF version of the text to be specified.
When \package{hyperref} is \emph{not} loaded, this command is defined so it may always be used in the document.

%-------------------------------------------------------------------------------
\section{An Example with \thispackage}\label{section:example}

\lstinputlisting[frame=none]{example/thesis.tex}

%-------------------------------------------------------------------------------
%\section{Release Notes}\label{section:release notes}

\end{document}